In condensed matter physics we study the behavior of crystals at finite density and low temperatures. By tuning and breaking the various materials, symmetries, and the topology of a crystal one can bring about brand new quantum phases of matter. These new phases of matter in turn produce emergent quasiparticles such as the cooper pair in superconductivity, the spinon in magnetic systems, and the Fermi arcs in Weyl semimetals. \par
Of particular interest are systems in which superconductivity interacts with topology. These systems have been theoretically predicted to produce anyonic quasiparticles which may be used as qubits in a future fault-tolerant quantum computer. However, these system usually require the use of the superconducting proximity effect to inject cooper pairs into the topological system. This is turn requires interfacing two different materials which not only requires extremely clean interfaces, but also matching Fermi surfaces, comparable Fermi velocities, and more. The ideal candidate for topological superconductivity would therefore be a material that is both superconducting and topologically non-trivial. One promising candidate is the iron-based superconductor FeTe$_{(1-x)}$Se$_{x}$, specifically the at \ac{FTS} doping  which also has non-trivial topology. In this dissertation we address the fabrication of pristine interfaces using a new tool as well as new probes into the topology of \ac{FTS}.\par
In Chapter II we discuss the motivation, construction, and use of the ``cleanroom-in-a-glovebox". 
This tool places an entire nanofabrication workflow into an inert argon atmosphere which has allowed us access to study a myriad of new materials and systems. A delightful offshoot of this glovebox is that it is a useful tool in training new scientists in fabrication techniques. The photolithography, \ac{PVD}, and characterization tools in the glovebox are designed to be easy to use and thus afford new users a low-risk method of learning new techniques.\par
In chapter III we discuss a specific example of a new quantum phase of matter e.g. topological superconductivity in FTS. There, I discuss the fabrication requirements to probe this elusive phase as well as the unique measurement technique used to provide evidence that FTS is a higher-order topological superconductor. The characterization of FTS continues in Chapter IV where we reveal some exciting new results in the \ac{FTS} system. These new results are direct evidence for the topological nature of \ac{FTS}, a feat which has only been shown in \ac{ARPES} and \ac{STM}\par
Chapter V concludes the dissertation with a summary of Chapters II, III, and IV. In addition, we give suggestions for future experiments to investigate the FTS system further as well as suggestions for insightful teaching programs with the cleanroom-in-a-glovebox.