\section{Summary}
In this dissertation we have presented numerous works investigating the topological nature of iron-based superconductor \acl{FTS}. To accomplish this, we required a fabrication environment that consistently left the surfaces and edges of the crystals in pristine condition. With this in mind, in Chapter II we discussed the concept, development, and optimization of the ``Cleanroom-in-a-Glovebox". This glovebox takes the workflow from a standard cleanroom photolithography process and condenses it into an inert argon environment. The merits of such works were demonstrated by comparing the Raman signals of various air-sensitive materials before and after exposure to air (Fig \ref{fig:FabricationFigure}). Furthermore, these materials were subsequently fabricated into electronic devices using photolithography and shown to have high quality Raman signals, demonstrating the power of the glovebox fabrication process. The linear layout of the fabrication workflow was optimized to maximize the number of available machines as well as minimize the time between fabrication steps. We also demonstrated that in addition to its powerful fabrication and characterization abilities, the glovebox is also a perfect tool for training the next generation quantum workforce. The simple interfaces of the fabrication facilities provide a low-stress situation for scientists to learn nanofabrication without fear of breaking the equipment. The conveyor-belt layout of the glovebox takes the mental load off of the student-scientists so they can focus on the creative and fun aspects of creating mesoscopic devices. Such considerations have already produced fantastic publications and collaborations from scientists of all experiences including summer scientists who were able to make significant contributions to ongoing lab work\cite{Gray2019, KUMAR2020112123, Gray2020} and collaborators who used the system to great effect answering fascinating questions\cite{WangBalgley2020, Kumar2020, lei2019high, Brotons-Gisbert2019, YaoCGT2Dmat}.\par
In Chapter III we presented strong evidence for a normal mode that exists purely on the hinge or the side of the \ac{FTS} crystal in the superconducting state. Recent theoretical work has suggested that such a mode could be the result of the combination of an exotic $s^{\pm}$ order parameter and a topological surface state. In short, the anisotropy of the superconducting phase gaps out adjacent faces of the \ac{TI} causing a normal mode at zero energy to appear at the hinge between the top and side surfaces. To observe this we fabricated a normal-metal / superconductor junction along the hinge of the material along with a built-in control junction which only contacts the c-axis of the material with the intent to investigate the \ac{DoS} of the hinge. When performing differential conductance measurements across the control junction a normal Andreev spectrum is observed that is consistent with point-contact measurements made on thin films of \ac{FTS}. In stark contrast, when the differential conductance measurements were made on the hinge contact an enormous zero-bias conductance peak emerges right below the critical temperature and does not seem to saturate as the temperature is lowered. This is strong evidence of a normal state emerging along either the hinge or side of the material that is not present along the c-axis of the material.\par
Finally, in Chapter IV we extended the scope of our electronic spectroscopy of \ac{FTS} to uncover exciting underlying physics. Specifically, we used cleaner crystals and lower temperatures to observe \acl{PAR} in \ac{FTS} indicating an underlying topological mechanism that allows tunneling carriers to ignore the potential barrier. Interestingly, the \ac{PAR} only manifests along clean, straight edges not along ``rough" edges indicating that this mechanism is either very sensitive to local conditions or its signal drowned out due to better contact to the superconducting bulk. To further investigate if this mechanism is topological in nature, we explored the response of the \ac{PAR} to external magnetic fields parallel to the $a$ and $c$-axis. We found that while the \ac{PAR} is quenched far quicker in a c-axis magnetic field than an a-axis magnetic field, it is not out of reach to attribute this to the bulk superconducting gap becoming quite small in a c-axis field, rather than the magnetic field affecting the \ac{PAR}-causing mechanism directly. Further measurements still need to be done to determine if a b-axis field affects the \ac{PAR} differently from an a-axis field. If it does, then we may be able to use the superconducting Doppler effect to determine where the tunnel junction is and if it is due to a 1D mode along the hinge or a 2D surface along the side of the material.

\section{Future Work}
Beyond finishing the work put forth in Chapter IV, there is a clear and exciting road forward for \ac{FTS} and other topological superconductors. Here I lay out some experiments I believe would provide interesting insight into the fundamental physics of \ac{FTS}.
\subsection{FeTeSe}
In Chapter IV we touched on the concept of symmetries protecting the topology of a crystal. This work provided evidence that there may be an important symmetry on the [100] face of the \ac{FTS} crystal but it would be useful to elucidate whether this is actually due to the c4 symmetry, if the contact to the superconducting bulk is better along rough edges, or if it's simply a matter of crystal quality. Here I suggest performing a series of three-point measurements (as shown in Chapters III and IV) on a variety of crystal facets: specifically on the [100], [110], and [010] facets as these would provide the strongest implication of the c4 symmetry. To accomplish this, one would naturally also need a new fabrication process to cut the exfoliated crystal into pristine edges. Here we suggest either RIE (in the fashion of graphene heterostructure devices), FIB (such as the beautiful devices shown in the work of G. B. Osterhoudt\cite{Osterhoudt2019}), or shadow-mask techniques while growing MBE thin-films. Furthermore, if the crystal facet is found to be an essential ingredient to produce \ac{PAR} then an important next step would be to determine if the \ac{PAR} also evolves with crystal thickness as this would be a perfect test for topology. In principle, if the topological surface states on opposing faces are not well-separated then these states will hybridize and open a gap. Therefore there should be a clear transition with decreasing thickness, e.g., going from observing \ac{PAR} to not observing \ac{PAR} on similar edges.\par
Along the same line of thought, it would be quite useful to be able to gate these normal modes in and out of the Dirac cone. In principle, if one tracks the normalized conductance as a function of gate voltage, the conductance should plateau at $2G_{N}$ (where $G_{N}$ is the conductance across the same contacts but at high bias) while the chemical potential is in the Dirac cone and then start to vary outside of the cone. This would be a direct test of the topology of the band structure put forth in Chapter \ref{chap:chap2}. This is of course a heavily idealized prediction of the experiment as there are many bulk bands that cross the Fermi energy that may complicate the data, but to first order I would hope that the normalization process takes care of that. In addition, many preliminary experiments need to be done before this type of device can be realized. The most pertinent of which is to use graphene to contact \ac{FTS} to see if the \ac{PAR} is still observed and to make sure the graphene itself doesn't contribute interesting physics to the tunneling phenomena. One would also need to make sure the gating doesn't affect the superconductivity. In principle, the superconductivity should perfectly screen the gate but given the inhomogeneous nature of \ac{FTS} it would be good to check that gating doesn't destroy the superconductivity by pinching off superconducting paths.\par
Another interesting parameter to tune would be the Te-doping level of \ac{FTS}. The hypothesis is that this will produce a similar result to the thickness experiment mentioned above, i.e., a topological transition as one dopes from FeSe down to FeTe. Much of the explanation given for the topological nature is the widening of the bandwidth when FTS is doped with tellurium, thus there should be a clear transition in the \ac{PAR} response when measuring different dopings.\par
Lastly, experiments that delve deeper into the band structure of \ac{FTS} are of great importance as the underlying topology won't be truly understood until the band structure is understood. While electrical gating allows us to probe the band structure locally, it is not strong enough to delve deeply into the bulk bands. Along this line, I see two paths for LASE to probe the preported Dirac crossings far above the Fermi level. The first would be to dope the material using the powerful technique developed by Yiping Wang, \textit{et al.} wherein the authors demonstrated remarkable doping levels in graphene using proximity to $\alpha$-RuCl$_{3}$\cite{WangBalgley2020}. The second would be to excite carriers from the Fermi level up to these buried Dirac crossings using a laser with the appropriate wavelength. However there are many questions with this experiment and there is little theoretical work about this type of experiment on \ac{FTS}.
\subsection{Fe-Based Superconductivity}
Finally, performing similar experiments beyond \ac{FTS} is of the utmost importance for myriad reasons. First, while \ac{FTS} is a fantastic material it is not stoichiometric and thus the exact doping ratio of tellurium to selenium can vary not only from crystal to crystal, but also from flake to flake. As discussed previously, the topology of \ac{FTS} is expected to depend heavily on the Te/Se ratio therefore even small deviations from the (55,45) ratio can have devastating consequences on experiments. Second, following from the first point, the topology and superconductivity of \ac{FTS} seems to be quite fragile and thus observations of phenomena which are a direct result of the topology and/or superconductivity pose a problem for the repeatability of such experiments. As an example, a recent high-profile paper observed ``nearly-quantized" zero-bias peaks in vortices of \ac{FTS} consistent with \ac{MZM}, however such zero-bias peaks were only observed in 1 out of every 60 vortices\cite{Zhu2020, frolov2021}.\par
An interesting material to start off these ``beyond \ac{FTS} measurements" with would be LiFeAs as it is also an iron-based superconductor with hole and electron pockets at $\Gamma$ and $M$ respectively, it is topological, has a high $T_{c}$ (18 K) and $H_{c2}$ (80 T), and it is stoichiometric\cite{Nag2016, Zhang2019_2, Tapp2008}. In addition, its air-sensitivity should not be a problem for LASE due to the cleanroom-in-a-glovebox, thus LASE is perfectly suited for measurements of this material. 