\todo{Write summary of FTS and glovebox. Give ideas for new experiments.}
\section{Summary}
In this dissertation we have presented numerous works investigating the topological nature of iron-based superconductor \acl{FTS}. To accomplish this, we required a fabrication environment that consistently left the surfaces and edges of the crystals in pristine condition. With this in mind, in Chapter II we discussed the concept, development, and optimization of the ``Cleanroom-in-a-Glovebox". This glovebox takes the workflow from a standard cleanroom photolithography process and condenses it into an inert argon environment. The merits of such works were demonstrated by comparing the Raman signals of various air-sensitive materials before and after exposure to air. Furthermore, these materials were subsequently fabricated into electronic devices using photolithography and shown to have similar quality Raman signals, demonstrating the power of the glovebox fabrication process. The linear layout of the fabrication workflow was optimized to maximize the number of available machines as well as minimize the time between fabrication steps. In addition to powerful fabrication and characterization abilities, the glovebox is also a perfect tool for training the next generation quantum workforce. The simple interfaces of the fabrication facilities provide a low-stress situation for scientists to learn nanofabrication without fear of breaking the equipment. The conveyor-belt layout of the glovebox takes the mental load off of the student-scientists so they can focus on the creative and fun aspects of creating mesoscopic devices.\par
In Chapter III we presented strong evidence for a normal mode that exists purely on the hinge or the side of the \ac{FTS} crystal in the superconducting state. Recent theoretical work has suggested that such a mode could be the result of the combination of an exotic $s^{\pm}$ order parameter and a topological surface state. In short, the anisotropy of the superconducting phase gaps out adjacent faces of the \ac{TI} causing a normal mode at zero energy to appear at the hinge between the top and side surfaces. 
\par
Finally, in Chapter IV we extend the scope of our electronic spectroscopy probe of \ac{FTS} to undercover exciting underlying physics. Specifically, we used cleaner crystals and lower temperatures to observe \acl{PAR} in \ac{FTS}. Interestingly, the \ac{PAR} only manifests along clean, straight edges not along ``rough" edges. 

\section{Future Work}
Beyond finishing the work put forth in Chapter IV, there is a clear and exciting road forward for \ac{FTS} and other topological superconductors. Here I lay out some experiments I believe would provide interesting insight into the fundamental physics of \ac{FTS}.
\subsection{FeTeSe}
In Chapter IV we touched on the concept of symmetries protecting the topology of a crystal. This work provided evidence that there is an important symmetry on the [100] face of the \ac{FTS} crystal but it would be useful to elucidate whether this is actually due to the c4 symmetry. Here I suggest performing a series of three-point measurements shown in Chapters III and IV on a variety of crystal facets: specifically on the [100], [110], and [010] facets as these would provide the strongest implication of the c4 symmetry. To accomplish this, one would naturally also need a new fabrication process to cut the exfoliated crystal into pristine edges. Here we suggest either RIE (in the fashion of graphene heterostructure devices), FIB (such as the beautiful devices shown in the work of G. B. Osterhoudt\cite{Osterhoudt2019}), or shadow-mask techniques while growing MBE thin-films. Furthermore, if the crystal facet is found to be an essential ingredient to produce \ac{PAR} then an important next step would be to determine if the \ac{PAR} also evolves with crystal thickness as this would be a perfect test for topology. In principle, if the topological surface states on opposing faces are not well-separated then these states will hybridize and open a gap. Therefore there should be a clear transition with decreasing thickness, e.g., going from observing \ac{PAR} to not observing \ac{PAR} on similar edges.\par 
Another interesting parameter to tune would be the Te-doping level of \ac{FTS}. The hypothesis is that this will produce a similar result to the thickness experiment mentioned above, i.e., a topological transition as one dopes from FeSe down to FeTe. Much of the explanation given for the topological nature is the widening of the bandwidth when FTS is doped with tellurium, thus there should be a clear transition .\par 
Lastly, experiments that delve deeper into the band structure of \ac{FTS} are of great importance as the underlying topology won't be truly understood until the band structure is understood. Along this line, I see two paths to use this same electronic spectroscopy technique at 
\subsection{Fe-Based Superconductivity}
Finally, the above measurements should all be performed again with 