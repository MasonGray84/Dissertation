At the beginning of the 20$\textsuperscript{th}$ century the world underwent the first quantum revolution. New ideas about wave-particles duality and quantization gave scientists the tools to explain previously observed phenomena such as the periodic table. With a deeper understanding, this new quantum theory drove revolutionary technologies such as electronic semiconductors thus bringing the world into the Information age. Now we are undergoing a second quantum revolution where we are no longer using quantum mechanics to simply explain observed phenomena, we are actively \textit{controlling} quantum mechanics.\cite{Dowling2003} We are using quantum technologies to organize and build complex systems at the atomic level. This extraordinary leap forward has allowed us to create and research new quantum phases of matter and their associated new quasi-particles. Aside from the importance of understanding novel fundamental physics, research into new quantum phases of matter is paramount for driving new technologies forward. For example, research into high-temperature superconductivity may lead us to room-temperature superconductivity, a phenomena which would massively reduce energy dissipation in modern electronics. In this dissertation, we describe the development of new equipment to help build these new quantum tools as well as the novel iron-based topological superconducting system such equipment has allowed us to probe.
\section{Scope}
The works presented in this dissertation fall into two main parts: advances in nano-fabrication equipment and topological superconductivity. The first part introduces recent advances in condensed matter physics along with the difficulties associated with fabricating electronic devices to better study these new topics. In particular, we discuss materials that are acutely air-sensitive such as GdTe$_{3}$ as well as materials that are stable in air but have air-sensitive surfaces such as \ac{FTS}. The second part dives into the subject topological superconductors and higher order topology. Specifically, we focus on the iron-based superconductor \acl{FTS}, its topological properties, and some exciting new experiments.\par
The rest of this introduction will review pertinent background material. We introduce the notion of emergence and quantum phases of matter with specific applications to superconducting materials. We will leave some of the more advanced subjects of superconductivity, e.g. tunneling into a superconductor from a normal metal, to the appendices where these subjects can get a more in-depth treatment. A brief overview of topology will be given but more focus will be spent on how to treat the notion of topology in superconductivity.
\section{Emergence, New phases, and Quasi-particles}
Emergence can be colloquially summarized as, ``The whole is greater than the sum of its parts." Examples of emergence are all around us from the biggest of scales where galaxies coalesce into superclusters to the smallest of scales where atoms emerge out of the fundamental excitations of quantum fields. With such a huge subject it may be difficult to see how this concept is useful in driving a research direction. Thus to keep this work on track we will use the sharper definition provided by Kivelson \& Kivelson:
\begin{quote}
	``An emergent behavior of a physical system is a qualitative property that can only occur in the limit that the number of microscopic constituents tends to infinity."\cite{Kivelson2016}
\end{quote}
Using this definition, it is evident that emergence heavily influences condensed matter systems. Indeed, phases of matter are a prime example of emergence as a single crystal can exhibit wildly different properties depending on \textit{external} conditions. Many of these phases of matter can be described elegantly through the language of symmetry and symmetry breaking\cite{Noether1918, Landau1937, pathria_beale_2022}. The first example of such a phase was the Quantum Hall Effect where the 
\section{Topological Superconductivity}
