Importance of emergence and new phases of matter.
\section{Scope}
The works presented in this dissertation fall into two main parts: advances in nano-fabrication equipment and topological superconductivity. The first part introduces recent advances in condensed matter physics along with the difficulties associated with fabricating electronic devices to better study these new topics. In particular, we discuss materials that are acutely air-sensitive such as GdTe$_{3}$ as well as materials that are stable in air but have air-sensitives surfaces such as \ac{FTS}. The second part dives into the subject topological superconductors and higher order topology.\par
The rest of this introduction will review pertinent background material. We introduce the notion of emergence and quantum phases of matter with specific applications to superconducting materials. We will leave some of the more advanced subjects of superconductivity, e.g. tunneling into a superconductor from a normal metal, to the appendices where these subjects can get a more in-depth treatment. A brief overview of topology will be given but more focus will be spent on how to treat the notion of topology in superconductivity.
\section{Emergence, New phases, and Particles}
Emergence can be summarized as, ``The sum of the whole is greater than the sum of the parts." More rigorously emergence describes when a system is observed to have properties its parts do not have on their own; these properties only emerge when the parts interact in a wider whole. 
\section{Topology and Superconductivity}